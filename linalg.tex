
\title{Linear Algebra cheat sheet}
\author{
        Ellin Zhao
}
\date{\today}

\documentclass[11pt]{article}
\usepackage{amssymb}
\usepackage[shortlabels]{enumitem}
\usepackage{marginnote}
\usepackage{amsmath}
\usepackage[margin=1in]{geometry}

\newcommand{\Thm}{\fbox{Thm}}
\newcommand{\Def}{\fbox{Def}}
\newcommand{\Fact}{\fbox{Fact}}

\newcommand{\R}{\mathbb{R}}
\newcommand{\A}{\mathbf{A}}
\newcommand{\B}{\mathbf{B}}
\newcommand{\I}{\mathbf{I}}

\begin{document}
\maketitle

\section{Chapter 1}

\begin{itemize}
\item[\Thm]
Given $\A\vec{x} = \vec{b}$, $\A$ is $m \times n$, the following statements are logically equivalent: 
\begin{enumerate} [a)]
  \item For each $\vec{b}$ in $\R^m$, $\A \vec{x} = \vec{b}$ has a solution.
  \item Each $\vec{b}$ in $\R^m$ is a linear combination of columns of $\mathbf{A}$.
  \item The columns of $\A$ span $\R^m$.
  \item $\A$ has a pivot position in every row.
\end{enumerate}

\end{itemize}


\subsection{Transformations}

\begin{itemize}
\item[\Def]
A transformation $T$ is linear if:
\begin{enumerate} [a)]
  \item $T(\vec{u} + \vec{v}) = T(\vec{u}) + T(\vec{v}), \forall \vec{u}, \vec{v} \in T$  
  check!!
  \item $T(c \vec{u}) = c \, T(\vec{u}), \forall c \in \R, \vec{u} \in T$
\end{enumerate}

\item[\Def] A mapping $T:\R^n \rightarrow \R^m$ is \textbf{onto} $\R^m$ if each 
$\vec{b}$ in $\R^m$ is the image of at least one $\vec{x}$ in $\R^n$.
This is assuming the mapping is Ax=b.
\begin{itemize}
  \item pivots in every row iff columns span $\R^m$.
\end{itemize}

\item[\Def] A mapping $T:\R^n \rightarrow \R^m$ is \textbf{one-to-one} if each 
$\vec{b}$ in $\R^m$ is the image of at most one $\vec{x}$ in $\R^n$.
This is assuming the mapping is Ax=b.
\begin{itemize}
  \item pivots in every column iff columns are linearly independent.
\end{itemize}

\end{itemize}

\newpage

\section{Chapter 2}

\begin{itemize}

\item[\Thm] Matrix transpose, inverse identities:
 \begin{enumerate} [a)]
   \item $(\A^\intercal)^\intercal = \A$
   \item $(\A + \mathbf{B})^\intercal = \A^\intercal + \mathbf{B}^\intercal$
   \item $(r \, \A)^\intercal = r\A^\intercal, \forall r \in \R$
   \item $(\A \mathbf{B})^\intercal = \mathbf{B}^\intercal \A^\intercal$
   \item $(\A^{-1})^{-1} = \A$
   \item $(\A \mathbf{B})^{-1} = \mathbf{B}^{-1} \A^{-1}$
   \item $(\A^\intercal)^{-1} = (\A^{-1})^\intercal$
 \end{enumerate}
 
\item[\Thm] Inverse of a $2 \times 2$ matrix:
$$
\A = 
\begin{bmatrix}
    a & b \\
    c & d 
\end{bmatrix}
\qquad
\A^{-1} = 
\frac{1}{ad-bc}
\begin{bmatrix}
    d & -b \\
    -c & a 
\end{bmatrix}
$$
\end{itemize}

\newpage

\section{Chapter 3}

\begin{itemize}
\item[\Thm] Determinant properties
\begin{enumerate} [a)]
  \item Multiple of one row of $\A$ added to another to form $\mathbf{B}$, 
  $\text{det} \, \mathbf{B} = \text{det}\, \A$.
  \item Two rows of $\A$ interchanged to produce $\mathbf{B}$, 
  $\text{det}\, \mathbf{B} = - \text{det} \, \A$.
  \item One row of $\A$ multiplied by $k$ to produce $\mathbf{B}$,
  $\text{det} \, \mathbf{B} = k \, \text{det}\, \A$.
\end{enumerate}

\item[\Thm] If $\A_{n \times n}$, then $\text{det} \, \A^\intercal = \text{det} 
\, \A$.

\item[\Thm] If $\A_{n \times n}, \mathbf{B}_{n \times n}$, then 
$\text{det} \, (\A \mathbf{B}) = (\text{det} \, \A)(\text{det} \, \mathbf{B})$. 

\end{itemize}

\subsection{Cramer's Rule}

\begin{itemize}

\item[\Thm] $\A_{n \times n}$ and invertible. $\A_i(\vec{b})$ is made by replacing 
the $i^{th}$ column of $\A$ with $\vec{b}$. $x_i$ is the $i^{th}$ element of $\vec{x}$.

For any $\vec{b} \in \R^n$, $x_i = \frac{\text{det} \, \A_i(\vec{b})}
{\text{det} \, \A}, i=1,2,...,n$.
\end{itemize}

\newpage

\section{Chapter 4}

\begin{itemize}
\item[\Def] A \textbf{vector space} is a nonempty set of vectors for which the following 
ten axioms hold: note that u, v, w in V
\begin{enumerate} [1)]
  \item $\vec{u} + \vec{v} \in \mathcal{V}$
  \item $\vec{u} + \vec{v} = \vec{v} + \vec{u}$
  \item $(\vec{u} + \vec{v}) + \vec{w} = \vec{u} + (\vec{v} + \vec{w})$
  \item $\vec{u} + \vec{0} = \vec{u}$ (a $\vec{0}$ exists in $\mathcal{V}$)
  \item $\forall \vec{u} \in \mathcal{V}, \vec{u} + (-\vec{u}) = \vec{0}$
  \item $c \vec{u} \in \mathcal{V}$
  \item $c \, (\vec{u} + \vec{v}) = c \vec{u} + c \vec{v}$
  \item $(c+d)\, \vec{u} = c \vec{u} + d \vec{u}$
  \item $c (d \vec{u}) = (cd) \vec{u}$
  \item $1\vec{u} = \vec{u}$
\end{enumerate}

\item[\Def] A \textbf{subspace} of $\mathcal{V}$ is a subset $\mathcal{H}$ of $
\mathcal{V}$ with the following three properties:
\begin{enumerate} [a)]
  \item The $\vec{0}$ of $\mathcal{V}$ is in $\mathcal{H}$
  \item $\vec{u} + \vec{v} \in \mathcal{H}, \forall \vec{u}, \vec{v} \in \mathcal{H}$
  \item $c \vec{u} \in \mathcal{H}, \forall c \in \R, \forall \vec{u} \in \mathcal{H}$
\end{enumerate}

\item[\Thm] If $\vec{v_1}, ... , \vec{v_p} \in \mathcal{V}$, then 
Span\{${\vec{v_1}, ... , \vec{v_p}}$\} is a subspace of $\mathcal{V}$.

\item[\Def] Nul\,$\A$ = \{$\vec{x} : \vec{x} \in \R^n, \A \vec{x} = \vec{0}$\}

\item[\Def] Col\,$\A$ = \{$\vec{b} : \vec{b} = \A \vec{x}, \text{for some } \vec{x} \in \R^n$\}
\begin{itemize}
  \item pivot columns of $\A$ forms a basis for Col\,$\A$
\end{itemize}
  
\item[\Def] Suppose $\mathcal{H}$ is a subspace of $\mathcal{V}$. 
$\beta = \{\vec{b_1},...,\vec{b_p}\}$ in $\mathcal{V}$ is a \textbf{basis} for $\mathcal{H}$ if:
\begin{enumerate} [a)]
  \item $\beta$ is a linearly independent set
  \item $\mathcal{H} = \text{Span}\,\{\vec{b_1},...,\vec{b_p}\}$
\end{enumerate}

\item[\Def] rank\,$\A$ = dim Col\,$\A$ = dim Row\,$\A$ = dim Col\,$\A^\intercal$ \\
rank\,$\A$ + dim Nul\,$\A$ = $n$

\end{itemize}

\newpage

\section{Chapter 5}
\begin{itemize}

\item[\Thm] The eigenvalues of a triangular matrix are the entries on its main diagonal.

\item[\Thm] If $\vec{v}_1,...,\vec{v}_r$ are eigenvectors that correspond to distinct 
eigenvalues $\lambda_1,...,\lambda_r$, then the set \{$\vec{v}_1,...,\vec{v}_r$\} is 
linearly independent.

\item[\fbox{Fact}] det\,$\A$ = 
$\begin{cases} 
      (-1)^r \cdot (\text{product of pivots in echelon form}), & \A \text{ is invertible} \\
      0, & \A \text{ is not invertible} \\
   \end{cases}
$

\item[\Thm] Determinant properties: $\A_{n \times n}, \B_{n \times n}$
\begin{enumerate} [a)]
  \item $\A$ is invertible $\iff$ det\,$\A \neq 0$
  \item det\,($\A \B$) = (det\,$\A$)(det\,$\B$)
  \item det\,$A^\intercal$ = det\,$\A$
  \item If $\A$ is triangular, then det\,$\A$ = product of entries on main diagonal
\end{enumerate}

\item[\Fact] $\lambda$ is an eigenvalue of $A_{n \times n} \iff$ det\,($\A - \lambda \I$) = 0 

\item[\Def] Suppose $\A_{n \times n}, \B_{n \times n}$. $\A$ is similar to $\B$, 
denoted $\A \sim \B$, if $\, \exists \, \mathbf{P}$ such that 
$\mathbf{P}^{-1} \A \mathbf{P} = \B$.

\item[\Thm] If $\A \sim \B$, then they have the same characteristic polynomial and the same 
eigenvalues with the same multiplicities.

\item[\Thm] $\A_{n \times n}$ is diagonalizable $\iff \A $ has $n$ linearly independent
eigenvectors.

\item[\Thm] Suppose  $\A_{n \times n}$ with distinct eigenvalues: $\lambda_1,...,\lambda_p$
\begin{enumerate} [a)]
  \item For $1 \leq k \leq p$, the dimension of the eigenspace for $\lambda_k$ is less than
  or equal to the multiplicity of $\lambda_k$.
  \item $\A$ is diagonalizable $\iff$ sum of dimensions of the eigenspaces = $n$.
  \item If $\A$ is diagonalizable and $\beta_k$ is the basis for the eigensapce corresponding 
  to $\lambda_k$ for each $k$, then $\beta_1,...,\beta_p$ forms an eigenvector basis 
  for $\R^n$.
\end{enumerate}

\end{itemize}

\newpage 

\section{Chapter 6}
later!

\newpage 

\section{Chapter 7}

\newpage

\begin{itemize}
\item[\Thm] Invertible Matrix Theorem: Given $\A$ is $n \times n$, the following statements are logically equivalent:
\begin{enumerate} [a)]
  \item $\A$ is invertible.
  \item $\A$ is row equivalent to $\I_{n \times n}$.
  \item $\A$ has $n$ pivot positions.
  \item $\A \vec{x} = \vec{0}$ has only trivial solutions.
  \item Columns of $\A$ form a linearly independent set.
  \item $\A \vec{x} = \vec{b}$ has at least one solutions for each $\vec{b}$ in $\R^n$.
  \item $\vec{x} \mapsto \A \vec{x}$ is one-to-one.
  \item Columns of $\A$ span $\R^n$.
  \item $\vec{x} \mapsto \A \vec{x}$ maps $\R^n$ onto $\R^n$.
  \item $\exists \, \mathbf{C}_{n \times n}$ such that $\mathbf{C} \A = \I$.
  \item $\exists \, \mathbf{D}_{n \times n}$ such that $\A \mathbf{D} = \I$.
  \item $\A^\intercal$ is an invertible matrix.
  \item Columns of $\A$ form a basis of $\R^n$.
  \item Col $\A$ = $\R^n$.
  \item dim Col $\A$ = $n$.
  \item rank $\A$ = $n$.
  \item Nul $\A$ = ${\vec{0}}$ check this...
  \item dim Nul $\A$ = 0.
  \item 0 is not an eigenvalue of $\A$.
  \item The determinant of $\A$ is not 0.
\end{enumerate}
\end{itemize}

\end{document}
